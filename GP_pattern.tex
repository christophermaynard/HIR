\documentclass[a4paper,11pt]{report}
\usepackage{listings}
\usepackage{xcolor}

\lstdefinestyle{dsl}
{
  language=python, 
  keywordstyle=\color{red}, 
  commentstyle=\color{blue}, 
  morekeywords={forall},   keywordstyle=\color{red},        
  numbers=left, numberstyle=\tiny, stepnumber=1, numbersep=5pt,
  classoffset=1,
  morekeywords={computation, field, imap},   keywordstyle=\color{orange}
}

\begin{document}
\begin{lstlisting}[style=dsl]
computation GP():
  field rhs, f
  imap map1
  tmp fem_tmp

  forall cell in horizontal domain:
    for k in bottomLayer, topLayer-1:
      fem_tmp = {
            # FEM computation, 
            # loop nest over quadrature, ndf1, ndf2
            # evaluation is sum into fem_tmp
            # of sums of products of weights, 
            # basis_functions, field1
        }
      forall df in ndf1:
        rhs[map1[cell,df]+k]] = \ 
           rhs[map1[cell,df]+k] + fem_tmp[df]

\end{lstlisting}
\subsection{Dimensionality}
\begin{enumerate}
  \item Fields are one-dimensional unique degrees of freedom over an unstructured mesh.
  \item They are of a builtin type, a floating point number, which could, for example be double precision.
   \item The dimesionality is fixed, but is fields on different function spaces have different sizes.
   \item The computation can be parallelised over columns. 
   \item Some kernels may be parallelised over the vertical dimension.
\end{enumerate}

\subsection{Access patterns}
  \begin{enumerate}
     \item Dofs which live on horizontally shared mesh entities, such as a
       vertical face, will be accessed by multiple columns.
     \item Dofs which live on vertically shared mesh entities, such as
       horizontal face, will be accessd by multiple layers.
     \item In general, the application will access all the neighbours
     \item This kernel only access one level of neighbours.
     \item This kernel has uniform access.
     \item Accesses are absolute horizontally and relative vertically.
     \item Indirect addressing via the imap is required for the
       absolute horizontal addressing.
     \item The field f is {\em Read}, the field RHS is {\em incremented}.
  \end{enumerate}

\subsection{Indirect addressing}
  \begin{enumerate}  
     \item The imap array is a two dimensional array, where {\em cell} is the
       number of horizontal cells (columns) in the mesh and df is the
       number of degrees of freedom per cell. For example, if the
       field has dofs which live on faces, the ndf = 6 for
       quadrilateral elements.
     \item The imap array is of a builtin type, integers.
     \item In parallel, it is partitioned by the domain
       decomposition, {\em i.e.} over cells.
  \end{enumerate}
\end{document}
\documentclass[a4]{article}
\usepackage{listings}
\usepackage{xcolor}

\lstdefinestyle{dsl}
{
  language=python, 
  keywordstyle=\color{red}, 
  commentstyle=\color{blue}, 
  morekeywords={forall},   keywordstyle=\color{red},        
  numbers=left, numberstyle=\tiny, stepnumber=1, numbersep=5pt,
  classoffset=1,
  morekeywords={computation, field, imap},   keywordstyle=\color{orange}
}

\begin{document}
\begin{lstlisting}[style=dsl]
computation GP():
  field rhs, f
  imap map1
  tmp fem_tmp

  forall column in domain:
    for k in bottomLayer, topLayer-1:
      fem_tmp = {
            # FEM computation, 
            # loop nest over quadrature, ndf1, ndf2
            # evaluation is sum into fem_tmp
            # of sums of products of weights, 
            # basis_functions, field1
        }
      forall df in ndf1:
        rhs[map1[cell,df]+k]] = \ 
           rhs[map1[cell,df]+k] + fem_tmp[df]

\end{lstlisting}
\subsection{Dimensionality}
\begin{enumeration}
  \item Fields are one-dimensional unique degrees of freedom over an unstructured mesh.
  \item They are a builtin type, a floating point number, which could for example be double precision.
   \item The dimesionality is fixed, but is fields on different function spaces have different sizes.
   \item The computation can be parallelised over columns. 
   \item some kernels may be parallelised over the vertical dimension.
\end{enumeration}

\subsection{Access patterns}
  
\end{document}